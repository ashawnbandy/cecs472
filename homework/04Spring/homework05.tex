\input macros
\input cstuff
\headline{{\bf CECS 472\hfill Project 5 \hfill Spring 2004}}
\footline{Dennis Volper \hfill 25 February 2004 (Week 5 Lecture 2)\hfill 
Due: 8 March 2004 (Week 7 Lecture 1)}
\parindent 0pt

Purpose: This assignment requires an in-depth understanding of
a single process concurrent connection-oriented client-server program 
(Chapter 13).
You will build a multi-user talk client and server.
You will use a concurrent connection-oriented single process server 
(TCP/select).

{\it THE MUT SERVER}

The MUT server will listen on your well-known TCP port.
The server will accept up to as many client connections at a time as is
allowed by the number of file descriptors.
Whenever it receives a message from a client it will echo that message
to all other clients currently connected (but not to the original sender).

The server will use select so that it gets both incoming
messages and incoming connection requests.
It will echo the incoming messages using a for loop, containing
an if statement that carefully skips the master socket, the socket of the
client that sent the message and all closed sockets.

The server must close any socket that it gets
an end of file or error signal on ({\ltt{}<= 0}).
Such a socket must be removed from the list of active file descriptors.
To to this you are required
to use the {\ltt{}send} and {\ltt{}recv}
system calls (instead of {\ltt{}read} and {\ltt{}write} so that
the extra ``flags" parameter is available.
by setting the {\ltt{}MSG_NOSIGNAL} flag
in your {\ltt{}send} and {\ltt{}recv} system calls.
%to disable the {\ltt{}SIGPIPE} signal
%so that on a broken write/read a -1 is returned instead of a {\ltt{}SIGPIPE}
%being thrown.
%You can do this by using the {\ltt{}signal} command to set the handler for {\ltt{}SIGPIPE}
%to the value {\ltt{}SIG_IGN}.

The server is required  report (log) the following pieces of information to
the screen (it's own screen) to make it easier for the
instructor to grade the assignments.
First, when a client connects, 
the server will report the file descriptor
returned by the accept, and the IP number of the client
in the form\hfill\break
{\ltt{}connect: 7 134.139.45.6}
\hfill\break
Second, when a client disconnects/dies, the server will report the file
descriptor (that is being closed
in the form\hfill\break
{\ltt{}disconnecting: 7}
\hfill\break
In addition, if a client disconnects (as opposed to dies), the
server will send the ``disconnecting: 7" string as a final message message
to the client. Hint, use {\ltt{}sprintf} like in the browser.
\hfill\break
Third, the server receives a message it should print the client's
socket number and the message.

{\it THE MUT CLIENT}

When started, the client will open a TCP connection to the server.
Whatever the client sees the user type will be sent to the server 
(i.e., the client gets a line from the keyboard, 
then sends that line to the server).
Whatever the client recvs from the server will be printed on the screen
(i.e., the client recvs from the socket, 
then prints what it has recv'd to the screen).
Typing control-D (EOF from keyboard) should cause the client to 
shutdown the socket, recv from the socket until it gets an end of file 
(there will be at least the final ``disconnecting" message from the
server which the client should receive and print), then exit.
Since input comes from both the keyboard and the socket your client
must use select.
Note: both see Comer's {\ltt{}TCPecho.c} for EOF handling.
Remember: {\ltt{}NULL} equals {\ltt{}false} in {\ltt{}C}.

This assignment requires you to detect EOF from the keyboard.
If you use {\ltt{}fgets} (as you did in the last assignment),
read the manual entry to discover how to detect EOF.

{\bf Submit:} The a fully commented copy (print-out) of the source code for the 
client and the server.
In addition,
the source code for the client must be placed in your home directory
in a file named {\ltt{}mutc.c} and
the source code for the server must be placed in your home directory
in a file named {\ltt{}mutd.c}.

{\bf Discussion:}

The echo server ({\ltt{}TCPmechod.c}) of chapter 13 
(from my {\ltt{}comer_examples} directory)
is a good starting place.
The echo server sends the message back to the client from which it came.
You must send it back to all other clients.
This means placing the {\ltt{}send} within a loop, and an {\ltt{}if}.
Remember to check for errors.
A dead client should not cause the server to die.
If you get a send/receive error when talking to a client;
close the socket, clear the bit in the {\ltt{}afds} set and
{\it keep going} (don't exit or crash).
For logging, the internet address is found in the {\ltt{}fsin} structure
just after the accept.
You should do an {\ltt{}inet_ntoa} on the 4 byte ip number field.
See the primitive client in chapter 5 for the exact name of this field.
If the server dies, the client should exit.
Note: {\ltt{}TCPmechod.c} and {\ltt{}TCPecho.c} have been updated
recently to use {\ltt{}send} and {\ltt{}recv}, make sure you start
with my latest versions of them.
\vfill

Sample Interaction:

We assume four clients will be started from four different terminals. 
These terminals may be logged into different machines.
We will call these clients A, B, C, D.

\medskip

{\parskip=0pt\parindent=-1em\leftskip=1em

1) Client A starts (The server creates the connection. At this point anything
that is typed on client A has no place to go.)
Server reports ``connect: 5 134.139.45.6"

2) Client B starts.
Server reports ``connect: 6 134.139.45.17"

3a) Client B's user types ``Hi".

3b) Client A gets ``Hi" and displays it on the screen.

3c) Server reports ``6 Hi"

4) Client C starts.
Server reports ``connect: 7 134.139.45.1"

5a) Client A's user types ``Cold".

5b) Client B gets ``Cold" and displays it on the screen.

5c) Client C gets ``Cold" and displays it on the screen.

5d) Server reports ``5 Cold"

6) Client A's user types control-D; so Client A shuts down its socket.
The server receives the EOF from client A so the server 
reports ``disconnecting: 5" to the screen and also sends 
that message to Client A.
Client A gets from the server: ``disconecting: 5" and prints that message,
then it gets end of file server, so it exits.

7a) Client B's user type ``Hot".

7b) Client C gets ``Hot"

7c) Server reports ``6 Hot"

9) Client D starts.
Server reports ``connect: 5 134.139.45.4"

}
\par

Note: The emphasis is on the client server, not the user interface.
{\it Keep the user interface simple.}
In real talk programs the sender of each message is identified, do not do this.
In real talk programs either the screen is split or incoming messages are
not allowed to interrupt a line the user is typing, do not do this.
\bye
