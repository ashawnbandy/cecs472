\input macros
\input cstuff
\headline{{\bf CECS 472\hfill Project 7 \hfill Spring 2004}}
\footline{Dennis Volper \hfill 22 March 2004 (Week 9 Lecture 1)\hfill
Due: 31 March 2004 (Week 10 Lecture 2)}
\parindent 0pt

Purpose: To gain an in-depth understanding of
remote procedure calls.

You will build a remote program and a client that makes remote procedure calls 
to that program.
Your remote procedure calls will use TCP.

Your remote program (server) will provide several remote procedures.
Your rpc client will allow the user to call any of these procedures.
For the remote program you must use your personal RPC program
number.
Compile and run the {\ltt{}Uniq_Id.c} program 
(from {\ltt{}~volper/classes/472/programs/rpc_example})
to get your program number.

Overview: You are writting an rpc (client/server) program for an automobile
repair shop. When a customer brings a car for repair, information about
that car is entered into a remote data base.
This data base represents the queue of cars awaiting repair. 
When a technician becomes available, a car from the queue is given to that 
technician for repair (and that car is removed from the queue).

{\it THE SINGLE MACHINE PROGRAM} 

In accordance with the RPC development procedure covered in class a
single process (non-network, non-rpc) version of the program must be
build first.
Use the {\ltt{}repair.c} program found in the
{\ltt{}programs/rpc} directory
You should copy that program, compile and run it so you know what it does.
To build this into a rpc program you must to separate it into two files,
one file must contain the service procedures
and the other file must contain the user interface.
When building the RPC version of the program you must
organize the inteface procedures so you retain the same interface
for the services as in {\ltt{}repair.c}

{\it The Service Procedure}

You must make each of the services provided by the program:
( {\ltt{}QueueCar}, {\ltt{}RepairCar}, {\ltt{}Summarize},
{\ltt{}WaitingTime}, {\ltt{}CountCars}, {\ltt{}ClearGarage},
{\ltt{}GarageEmpty} )
available as remote procedures in a remote program.

{\it The User Interface}

The emphasis here is on the rpc implementation.
Keep the same (simplified) user interface provided by {\ltt{}repair.c}.

\vfill\eject

{\it THE RPC SERVER PROGRAM}

The rpc server program will be called {\ltt{}repaird} when running.

Make sure the number of values passed to each of the remote procedures
corresponds to the number of parameters the corresponding
sevice procedures requires.
Make sure the number of values returned by each of the remote procedures
corresponds to the number of values the corresponding
sevice procedures delivers back to the user.

Your job is to implement the server and the
the server-side interfaces to provide the correct packing and unpacking
necessary between the server procedures and the rpc calls.

{\it THE RPC CLIENT}

The rpc client program will be called {\ltt{}repairc} when running.
It will take one parameter, the name of the host to which it is to make
its rpc calls.
Example: {\ltt{}repairc cheetah}

Your job is to implement the client and the
the client-side interfaces to provide the correct packing and unpacking
necessary between the client and the rpc calls.
You must also make the appropriate modification to the main program.

{\bf Submit:} The fully commented copy (print-out) of the source code
for the following files:
{\ltt{}repair.x},
{\ltt{}repair_cif.c},
{\ltt{}repair_sif.c},
{\ltt{}repair_client.c},
{\ltt{}repair_services.c}.
In addition the source code for these files should be placed in your home
directory.
Note: {\ltt{}repair_client.c} and {\ltt{}repair_services.c} are going
to be mostly my code from {\ltt{}repair.c}.
I want to make sure you've made the appropriate modifications
and appropriately separated the service procedures.

Note that the files:
{\ltt{}repair_clnt.c},
{\ltt{}repair_svc.c},
{\ltt{}repair_xdr.c},
{\ltt{}repair.h},
will be automatically generated, but since you don't do anything to them
you should not submit them.

\bigskip
{\bf Reminder 1:}
Your parameters in 
{\ltt{}repair_client.c} and
{\ltt{}repair_services.c} {\it must}
use the types and modes specified for the single machine program.

{\bf Reminder 2:}
Remember that both {\ltt{}repair_client.c} and {\ltt{]repair_sif.c}
will need prototypes for each of the services functions.

\bye
