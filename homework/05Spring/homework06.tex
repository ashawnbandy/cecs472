\input macros
\input cstuff
\headline{{\bf CECS 472\hfill Homework 6 \hfill Spring 2005}}
\footline{Dennis Volper \hfill 9 February 2005 (Week 3 Lecture 1) \hfill
Due: 14 February 2005 (Week 3, Lab 2)}
\parindent 0pt

Purpose: This is designed to familiarize you with TCP clients.

Build a TCP time client in a file called {\ltt{}TCPtime.c}.
Your client will contact the ``special" time server located
on port 5001 of panther.

Use the main program--client procedure format.
The default service should {\ltt{}"5001"}.
The default host should be {\ltt{}"panther"}.

The time server will pass you back an 4 byte integer representing a 
UCT time in network standard order.
You will read that integer, convert it to host order,
and then UNIX epoch time,
then print it using the {\ltt{}ctime} system call.
For an example of how to convert and print the integer
see Comer's {\ltt{}UDPtime} client.

If the read fails, or the connection closes before you have
received 4 bytes; print an error message and exit.
Do not call {\ltt{}ctime}.

You don't have to handle the case of ``can't make a connection"
because Comer's code will print an error message and exit for you.

Discussion:
One of the points of the TCP discussion was that, in a stream,
boundaries are not preserved.
You will need to assemble the 4 bytes (since they may come in on
separate reads)
before doing the conversion.
The {\ltt{}panther} server is perverse and will try to force
the bytes to arrive at your client in separate reads.
It will occasionally lose (close) the connection (1 time in 5).

Submit: a printout of your {\ltt{}TCPtime.c} file.

A copy of this file must be located in your class home directory;
and have the name above.
This is to allow the instructor to use a script to test your homework.

Testing: what you should see is something fairly close to
the current date.
About 1 time in 5 you should see the error message.

\bye
