\input macros
\input cstuff
\headline{{\bf CECS 472\hfill Project 13 \hfill Spring 2005}}
\footline{Dennis Volper \hfill 14 March 2005 (Week 8 Lecture 1)\hfill 
Due: 16 March 2005 (Week 8 Lecture 2)}
\parindent 0pt

Purpose: This assignment uses the multiprotocol technique
topics covered in Chapter 14.
You will build a (multi-user) gettimeofday server
using the technique from Chapter 14.
The gettimeofday server delivers 8 bytes of time (seconds and microseconds)
on both TCP and UDP on your well-known-port.

For the exact UDP specification see homework 7.
For the exact TCP specification see homework 8.

Recommendation: start with a copy of Comer's {\ltt{}daytimed.c}
(the Chapter 14 example)
and merge in the appropriate parts of your {\ltt{}timed.c} from homework 7
and of your {\ltt{}Ttimed.c} from homework 8.

{\bf Submit:} The a fully commented copy (print-out) of the source code for the 
server.
In addition,
the source code for the server must be placed in your home directory
in a file named {\ltt{}UTtimed.c}.
Also, be sure that the files {\ltt{}timec.c} and {\ltt{}Ttimec.c} you
submitted for homeworks 7 and 8 respectively, are still in your home directory.
I will be using your clients when testing your server.
Do not submit copies of these clients (you already did that).

Note: Most of this assignment should be cut and paste.
Very little new code is necessary.
\bye
