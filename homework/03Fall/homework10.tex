\input macros
\input cstuff
\nopagenumbers
\headline{{\bf CECS 472\hfill Homework 10 \hfill Fall 2003}}
\footline{Dennis Volper \hfill 19 November 2003 (Week 12 Lecture 2)\hfill
Due: 1 December 2003 (Week 15 Lecture 1)}
\parindent 0pt

% Note, previously issued Week 13 Lecture 1

Purpose: Explore Routing, Domain Name Service and Subnetting Utilities

Submission: 
For the network topology question you will submit a picture which may be hand 
drawn. 
For all other you will submit short answers.
Remember, your analysis of the output of the command is an important part
of the assignment.

1)
Data link layer.
(Manual entries: arp, ping,)

Give part of the arp table for jaguar.
The jaguar arp table is huge so you will give the arp entries
You must include the arp entries for cheetah, panther, and lab37.

ping is used only to make sure the arp table cache has the entries you want.

\vfill
2) Interface configuration. (Manual entries: ifconfig)

For each IP interface give the following:
a) The IP number.
b) The name of the interface
c) The netmask.
d) The broadcast address

Answer questions (a), (b), (c) and (d) for each of the following machines:
\hfil\break
i) cheetah
\hskip1in
ii) jaguar
\hfil\break

\vfill
3) Routing tables. (Manual entries: netstat)

a) What cables is the machine connected to.
b) What gateways does the machine know about.

Answer questions (a), and (b) for each of the following machines:
\hfil\break
i) cheetah
\hskip1in
ii) jaguar
\hskip1in

\vfill
4) Remote access of data.
(Manual entry: snmpnetstat, snmpstatus)

The College of Engineering gateways have multiple interfaces.
One of these gateways has an interface with the IP address 134.139.248.25,
another has 134.139.248.1, another has 134.139.147.129.
a) For each of these gateways list the type of gateway and the
number of interfaces that are ``down" (this is the status of
the gateway).
b) For each of these gateways list
the ``IP Interfaces" that are currently active on these gateways
their internet numbers and their network masks (in /bits format).
Notes: Whereas cheetah uses eth0 and eth1 and the
gateways use things like et12, dhcp34 and FastEthernet0/12 for
their interface names.
{\ltt{}cheetah} supports snmpnetstat and snmpstatus.
The gateways will not report out their routing tables,
but it will respond to the {\ltt{}-i} option of {\ltt{}snmpnetstat}
that queries it's interfaces (which gives you the answer you need).
The community is required (as indicated in the manual entry)
eventhough the ``usage" the program prints indicates it isn't.

Remember: each port has {\it one} IP number, obviously
one of the ports has the IP number given above.

Note: entries without IP numbers are ``virtual" LANs and should
be ignored.
Be careful not to mix up the cablenumber/netmask with the interface
IP number.

5) Network topology. 
(Manual entries: traceroute)

Draw a map of the ``interesting" part Internet.
Include the following machines in your map and include all gateways
that connect those machines.
Your map should include IP numbers of as many of the network connections
as possible.
Using the information from questions 1-4 above, you should be able
to get IP numbers for all network connections.

\medskip

{\obeylines\parskip=0pt

panther.net.cecs.csulb.edu
jaguar.net.cecs.csulb.edu
cheetah.cecs.csulb.edu
aardvark.cecs.csulb.edu
ecs412lin67.cecs.csulb.edu
heart.cecs.csulb.edu
lab71.net.cecs.csulb.edu
lab37.net.cecs.csulb.edu
ecs416lin196.cecs.csulb.edu
charlotte.cecs.csulb.edu
134.139.59.11

}
\bye
