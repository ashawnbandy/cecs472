\input macros
\input cstuff
\headline{{\bf CECS 472\hfill Project 7 \hfill Fall 2002}}
\footline{Dennis Volper \hfill 28 October 2002 (Week 9 Lecture 1)\hfill
Due:6 November 2002 (Week 10 Lecture 2)}
\parindent 0pt

Purpose: To gain an in-depth understanding of
remote procedure calls.

You will build a remote program and a client that makes remote procedure calls 
to that program.
Your remote procedure calls will use TCP.

Your remote program (server) will provide three remote procedures.
Your rpc client will allow the user to call any of these procedures.
For the remote program you must use your personal RPC program
number.
Compile and run the {\ltt{}Uniq_Id.c} program 
(from {\ltt{}~volper/classes/472/programs/rpc_example})
to get your program number.

Overview: You are writing an rpc (client/server) program for a railroad freight
train.
When a shipment is loaded onto the train, information about
that shipment is entered into a remote data base.
This data base represents the shipments currently on the train.
When a shipment is unloaded (hopefully at its destination),
information about that shipment is returned to the client
and the entry is deleted from the remote database.

{\it THE SINGLE MACHINE PROGRAM} 

In accordance with the RPC development procedure covered in class a
single process (non-network, non-rpc) version of the program must be
built first. Use the {\ltt{}train.c} program found in the 
{\ltt{}programs/rpc} directory.
You should copy that program, compile and run it so you know what it does.
To build this into a rpc program you must to separate it into two files, 
one file must contain the service procedures 
and the other file must contain the user interface.
When building the RPC version of the program you must
organize the inteface procedures so you retain the same interface
for the services as in {\ltt{}train.c}.

{\it The Service Procedure}

You must make each of the services provided by the program:
( {\ltt{}LoadTrain}, {\ltt{}UnloadTrain}, {\ltt{}GetWeights},
{\ltt{}CountBoxes}, {\ltt{}IsTrainFull} {\ltt{}ClearTrain})
avaliable as remote procedures in a remote program.

{\it The User Interface}

The emphasis here is on the rpc implementation.
Keep the same (simplified) user interface provided by {\ltt{}train.c}. 

\vfill\eject

{\it THE RPC SERVER PROGRAM}

The rpc server program will be called {\ltt{}traind} when running.

Make sure the number of values passed to each of the remote procedures
corresponds to the number of parameters the corresponding
sevice procedures requires.
Make sure the number of values returned by each of the remote procedures
corresponds to the number of values the corresponding
sevice procedures delivers back to the user.

Your job is to implement the server and the
the server-side interfaces to provide the correct packing and unpacking
necessary between the server procedures and the rpc calls.

{\it THE RPC CLIENT}

The rpc client program will be called {\ltt{}trainc} when running.
It will take one command line argument, the name of the host to which 
it is to make its rpc calls.
Example: {\ltt{}trainc cheetah}

Your job is to implement the client and the
the client-side interfaces to provide the correct packing and unpacking
necessary between the client and the rpc calls.
You must also make the appropriate modification to the main program.

{\bf Submit:} The fully commented copy (print-out) of the source code
for the following files:
{\ltt{}train.x},
{\ltt{}train_cif.c},
{\ltt{}train_sif.c},
{\ltt{}train_client.c},
{\ltt{}train_services.c},
In addition the source code for these files should be placed in your home
directory.
Note: {\ltt{}train_client.c} and {\ltt{}train_services.c} are going 
to be mostly my code from {\ltt{}train.c}. 
I want to make sure you've made the appropriate modifications
and appropriately separated the service procedures.

Note that the files:
{\ltt{}train_clnt.c},
{\ltt{}train_svc.c},
{\ltt{}train_xdr.c},
{\ltt{}train.h},
will be automatically generated, but since you don't do anything to them
you should not submit them.

\bigskip
{\bf Reminder:}
Your parameters in 
{\ltt{}train_client.c} and
{\ltt{}train_services.c} {\it must}
use the types and modes specified in the single machine program.

\bye
