\input macros
\input cstuff
\headline{{\bf CECS 472\hfill Homework 8 \hfill Fall 2011}}
\footline{Dennis Volper \hfill  26 September 2011 (Week 5 Lecture 1) \hfill
Due: 28 September 2011 (Week 5, Lab 2)}
\parindent 0pt

Purpose: This assignment is to gain an in-depth understanding of
a TCP client-server program.

You will build a connection-oriented time client and an iterative
connection-oriented time server using the techniques found in Chapter 10
of the text. (You will use TCP.)

Submission: You must submit a print out of both the client and the server.
In addition you must place the source code for both the client and
the server in your home directory in files called {\ltt{}Ttimec.c}
and {\ltt{}Ttimed.c}, respectively. I will access these files to check
the correctness of your programs.

Port numbers: As in the previous assignment, you must use your personal
well-known-port number from the {\ltt{}get_port} procedure.

\noindent
{\it THE TIME SERVER}

Your time server will listen on your well known TCP port.
Whenever it accepts a connect from a client it will call {\ltt{}gettimeofday}
and return the {\ltt{}tv_sec} and {\ltt{}tv_usec} values to the client.
You are returning 8 bytes; you will need an 8 byte array.

Sequence:

1) Get a request.
\hfill\break
2) Use the {\ltt{}gettimeofday} call.
\hfill\break
3) Print (to the screen) the values of
{\ltt{}tv_sec} and {\ltt{}tv_usec} in hexadecimal.
\hfill\break
4) htonl the {\ltt{}tv_sec} and {\ltt{}tv_usec},
go ahead and put them right back into the structure.
\hfill\break
5) Use two calls to {\ltt{}memcpy}, to copy the sec into
the array starting at position 0 and
the usec into the array starting at position 4.
\hfill\break
6) Send the array (8 bytes).

\noindent
{\it THE TIME CLIENT}

Your time client will connect to server and get the time on the server's 
machine.
The client will connect (it will NOT send a message) and receive as a
reply the time on the server's machine given as the two {\ltt{}gettimeofday}
numbers ({\ltt{}tv_sec} and {\ltt{}tv_usec}).

You are receiving 8 bytes; you will need an 8 byte array.

You are receiving a time of day you are required to
use a {\ltt{}timeval} structure to contain the final answer.

Sequence:

1) Connect to the server.
\hfill\break
2) Read and reassemble the reply.
\hfill\break
3) Use two calls to {\ltt{}memcpy} to copy the sec and usec into
the ({\ltt{}tv_sec} and {\ltt{}tv_usec}) fields of a {\ltt{}timeval}
structure.
\hfill\break
4) ntohl the {\ltt{}tv_sec} and {\ltt{}tv_usec},
go ahead and but them right back into the structure.
\hfill\break
5) Print (to the screen) the values of
{\ltt{}tv_sec} and {\ltt{}tv_usec} in hexadecimal.

\bigskip
Recommendations: (not requirements)

Server)
This is a merge of the TCP server from Chapter 10
and your time server from the previous homework.
A good starting point is Comer's code,
copy in the time, htonl and memcpy from your previous homework.

Client) 
This is a merge of your UDP client from your homework
and your TCP reassembly loop from the TCP client homework.
Start with your TCP client from your homework, 
copy in the memcpy and htonl stuff after the
reassembly loop. Remember, it's 8 bytes this time
so change your array size and loop constant.
\bye
